\documentclass{article}
\usepackage[utf8]{inputenc}
\usepackage[margin=0.5in]{geometry}
\usepackage{amsmath}
\usepackage{amssymb}
\usepackage{physics}
\usepackage{ulem}
\usepackage{enumitem}
\usepackage{graphicx}

%Text   
\renewcommand{\t}[1]{\text{#1}}
\renewcommand{\it}[1]{\textit{#1}} 
\renewcommand{\bf}[1]{\textbf{#1}} 

%Calculus 
\newcommand{\s}[2]{\sum_{#1}^{#2}}
\renewcommand{\l}[2]{\lim_{{#1}\to{#2}}}
\renewcommand{\b}[1]{\left(#1\right)}
\renewcommand{\i}[4]{\int\limits_{#3}^{#4}{#1}\,\mathrm{d}{#2}}
\renewcommand{\d}{\mathrm{d}}


%Sconce
\newcommand{\sn}[2]{#1\times 10^{#2}}
\newcommand{\p}{\psi}
\renewcommand{\P}{\Psi}
\newcommand{\px}{\psi(x)}
\newcommand{\pxt}{\psi(x,t)}
\newcommand{\Px}{\Psi{x}}
\newcommand{\Pxt}{\Psi(x,t)}
\renewcommand{\Px}{\Psi}

% Linear Algebra
\renewcommand{\v}[2]{\begin{bmatrix} #1 \\ #2 \end{bmatrix}}
\newcommand{\vv}[3]{\begin{bmatrix} #1 \\ #2 \\ #3 \end{bmatrix}}
\newcommand{\vn}[3]{\begin{bmatrix} #1 \\ #2 \\ \   vdots \\#3 \end{bmatrix}}

\renewcommand{\r}[2]{\begin{bmatrix} #1 & #2 \end{bmatrix}}
\newcommand{\rr}[3]{\begin{bmatrix} #1 & #2 & #3 \end{bmatrix}}
\newcommand{\rn}[3]{\begin{bmatrix} #1 & #2 & \cdots &#3 \end{bmatrix}}

\renewcommand{\pm}[4]{\begin{pmatrix} #1 & #2 \\ #3 & #4\end{pmatrix}}
\newcommand{\bm}[4]{\begin{bmatrix} #1 & #2 \\ #3 & #4\end{bmatrix}}
\renewcommand{\d}[4]{\begin{vmatrix} #1 & #2 \\ #3 & #4\end{vmatrix}}

\newcommand{\pmm}[9]{\begin{pmatrix} #1 & #2 & #3 \\ #4 & #5 & #6 \\ #7 & #8 & #9 \end{pmatrix}}
\newcommand{\bmm}[9]{\begin{bmatrix} #1 & #2 & #3 \\ #4 & #5 & #6 \\ #7 & #8 & #9 \end{bmatrix}}
\newcommand{\vmm}[9]{\begin{vmatrix} #1 & #2 & #3 \\ #4 & #5 & #6 \\ #7 & #8 & #9 \end{vmatrix}}




\title{\bf{Quantum Mechanics Griffiths Notes}}
\author{Syed Khalid}
\date{October 2021}

\begin{document} 
\maketitle 
\section{Ch1 Wave Function}
\subsection{Schrodinger Equation}
$$i\hbar\pdv[]{\P}{t}=-\frac{\hbar^2}{2m}\pdv[2]{\P}{x}+V\P$$
Probability of finding particle between $a$ and $b$ at time $t$
$=\i{\abs{\Pxt}^2}{x}{a}{b}$

\subsection{Probability}

For a Discrete Variable $j$ :
\begin{align*}
\ev{j}=\frac{\sum jN(j)}{N_{total}}
=\s{j=0}{\infty}jP(j) \\
\ev{f(j)}=\s{j=0}{\infty}f(j)P(j)\\
\sigma=\sqrt{\ev{(\ev{j}-j)^2}}=\sqrt{\ev{j^2}-\ev{j}^2}
\end{align*}
For a Continuous Variable $x$:
\begin{align*}
    P(c\in[x,x+dx])=\rho(x)dx\\
    P(x\in[a,b])=\i{\rho(x)}{x}{a}{b}\\
    \i{\rho(x)}{x}{-\infty}{\infty}=1\\
    \ev{x}=\i{x\rho(x)}{x}{-\infty}{\infty}\\
    \ev{f(x)}=\i{f(x)\rho(x)}{x}{-\infty}{\infty}
\end{align*}
\subsection{Nomralization}

From the Born's Statistical Interpretation of $\P$:
$$\i{\abs{\Pxt}^2}{x}{-\infty}{\infty}=1$$
$$\dv[]{}{t}\i{\abs{\Pxt}^2}{x}{-\infty}{\infty}=0$$
Hence once the wave function is normalized at any $t$ its normalized for all $t$.
\subsection{Momentum}
$$\ev{x}=\i{\P^*[x]\P}{x}{D}{}$$
$$\ev{p}=\i{\P^*[-i\hbar\pdv{}{x}]\P}{x}{D}{}$$
\subsection{Uncertainty Principle}

$$\sigma_x\sigma_p\geq\frac{\hbar}{2}$$
\section{Time Independependent Schrodinger Equations}
\subsection{Stationary States}
Let $V(x)$ be independependent of time and $\Pxt=\px\varphi(t)$. From Schrodinger Equations we get the TISE:
$$\boxed{-\frac{\hbar^2}{2m}\dv[2]{\p}{x}+V\p=E\p},\,\,\,\,\dv{\varphi}{t}=-\frac{iE}{\hbar}\varphi\implies\varphi(t)=e^{-iEt/\hbar}$$
Where $\Pxt=\px e^{-iEt/\hbar}$ are stationary states as the probability density and every expectation value is independependent of time. 
$$\hat{H}=-\frac{\hbar^2}{2m}\dv[2]{}{x}+V(x), \,\,\,\, \boxed{\hat{H}\p=E\p},\,\,\,\ev{H}=E$$
TISE yields an infinite number of solutions each associated with an allowed energy.
 Any wave function can be written as a linear combination of these infinite stationary states:
 $$\Pxt=\s{i=0}{\infty}c_i\P_i(x,t)=\s{i=1}{\infty}c_i  \p_i(x)e^{-iEt/\hbar}$$
Where $\abs{c_i}^2$ represent te probability of the measurement of energy returning $E_i$. Thus:
$$\s{i=1}{\infty}\abs{c_i}^2=1,\,\,\, \ev{H}=\s{i=1}{\infty}\abs{c_i}^2E_i$$
\subsection{Infinite Square Well}
We definte $V(x)=0$ when $x\in(0,a)$ else, $V(x)=\infty$. TISE becomes:
$$\pdv[2]{\px}{x}=-(\frac{\sqrt{2mE}}{\hbar})^2\px$$
Resembling the Simple Harmonic Oscillator($\pdv[2]{f}{x}=-k^2f$). Applying boundary conditions $\p (0)=0,\,\,\,\p(a)=0$ and solving we get:
$$\p_n(x)=\sqrt\frac{2}{a}\sin(\frac{n\pi}{a}x),\,\,\,n=1,2,3\hdots$$
$$E_n=\frac{n^2\hbar^2\pi^2}{2ma^2}$$
$\p_n(x)$ are alternatively even and odd w.r.t to $x=a$. $\p_i(x)$ has $i-1$ nodes. Also they are orthonormal:
$$\i{\p_m(x)^*\p_n(x)}{x}{-\infty}{\infty}=\delta_{mn}$$ 
They are also complete:
$$\Pxt=\sqrt{\frac{2}{a}}\s{n=1}{\infty}c_nsin(\frac{n\pi}{a}x)e^{-it(n^2\pi^2\hbar/2ma^2)}
\\,\,\,\,c_n=\sqrt{\frac{2}{a}}\i{sin(\frac{n\pi}{a}x)\P(x,0)}{x}{0}{a}$$
This is nothing but the Fourier series.
\subsection{Harmonic Oscillator}
Defined by the potential- $V(x)=\cfrac{1}{2}m\omega^2x^2$. Most arbitary potentials can be expressed in this form;
$$V(x) = V(x_0)+V'(x_0)(x-x_0)+\cfrac{1}{2}V''(x_0)(x-x_0)^2+\dots$$
where $x_0$ is a minima. As
\subsubsection{Algebraic Method}

\subsubsection{Analytic Method}
\subsection{Free Particle}
\subsection{Delta-Function Potential}
\subsubsection{Bound States and Scattering States}
\subsubsection{Delta-Function Well}
\subsection{Finite Square Well}
$$ \bf{4.}\,\,\,\,\,\,\, V_1 = \cfrac{R_1}{R_1+R_2}\times V = \cfrac{10k}{10k+20k}5V = 1.67V$$
$$ \,\,\,\,\,\,\,\,\, V_2 = \cfrac{R_2}{R_1+R_2}\times V = \cfrac{20k}{10k+20k}5V = 3.33V$$

$$I_1 = \cfrac{R_2}{R_1+R_2}\times I = \cfrac{2k}{1k+2k}\times 15mA = 5mA$$
$$I_2 = \cfrac{R_1}{R_1+R_2}\times I = \cfrac{1k}{1k+2k}\times 15mA = 2.5mA$$

$$ C_{12} = \cfrac{C_1 \times C_2}{C_1+C_2} \implies \cfrac{20\text{mF}  \times 30\text{mF}}{20\text{mF}+30\text{mF}} = 12 \text{mF} $$
$$ C_{34} = C_3 + C_4 \implies 40 \text{mF} + 20 \text{mF} = 60 \text{mF} $$
$$ C_{total} = \cfrac{C_{12} \times C_{34}}{C_{12}+C_{34}} \implies \cfrac{12 \text{mF}\times 60\text{mF}}{12\text{mF}+60\text{mF}} = 10 \text{mF}$$
$$ Q = C_{total} \times V_{total} \implies 10\text{mF} \times 30 \text{V} = 300 \text{mC}$$
$$V_1 = \cfrac{Q}{C_1} \implies \cfrac{300 \text{mC}}{20\text{mF}} = 15\text{V} $$
$$V_2 = \cfrac{Q}{C_2} \implies \cfrac{300 \text{mC}}{30\text{mF}} = 10\text{V} $$
$$V_3 = \cfrac{Q}{C_{34}} \implies \cfrac{300 \text{mC}}{60\text{mF}} = 5\text{V} $$


$$    R_{total} = R_1+R_2+R_3 \implies 10k \Omega + 20k \Omega + 30k \Omega = 60k \Omega$$
$$   C_{total} = C_1+\cfrac{C_2\times C_3}{C_2+C_3} \implies 10 \text{nF}+ \cfrac{20\text{nF}\times 20\text{nF}}{20\text{nF}+20\text{nF}} = 20\text{nF}$$
$$    L_{total} = \cfrac{L_1\times(L_2+L_3)}{L_1+L_2+L_3} \implies \cfrac{10\text{mH}\times(4\text{mH}+8\text{mH})}{10\text{mH}+4\text{mH}+8\text{mH}} = 5.45 \text{mH}$$


\end{document}


