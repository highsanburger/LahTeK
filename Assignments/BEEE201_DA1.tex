\documentclass[14pt]{article}
\usepackage[utf8]{inputenc}
\usepackage[margin=0.5in]{geometry}
\usepackage{amsmath}
\usepackage{amssymb}
\usepackage{physics}
\usepackage{ulem}
\usepackage{enumitem}
\usepackage{graphicx}
\usepackage{blindtext}

\usepackage{draftwatermark}
\SetWatermarkText{21BEE1288}
\SetWatermarkScale{4}
\SetWatermarkLightness{0.96}

\usepackage{pgf}
\usepackage{pgfpages}
\pgfpagesdeclarelayout{boxed}
{
  \edef\pgfpageoptionborder{0pt}
}
{
  \pgfpagesphysicalpageoptions
  {%
    logical pages=1,%
  }
  \pgfpageslogicalpageoptions{1}
  {
    border code=\pgfsetlinewidth{1pt}\pgfstroke,%
    border shrink=\pgfpageoptionborder,%
    resized width=.95\pgfphysicalwidth,%
    resized height=.95\pgfphysicalheight,%
    center=\pgfpoint{.5\pgfphysicalwidth}{.5\pgfphysicalheight}%
  }%
}
\pgfpagesuselayout{boxed}

\title{\textbf{BEEE201L-Digital Assignment 3}}
\author{Syed Khalid}
\date{3rd June, 2022}
\begin{document}

\maketitle

\section{Liquid Crystal Display}%
  \label{sec:Liquid Crystal Display}
  \Large
 A liquid-crystal display (LCD) is a flat-panel display or other electronically modulated optical device that uses the light-modulating properties of liquid crystals combined with polarizers. Liquid crystals do not emit light directly, instead using a backlight or reflector to produce images in color or monochrome
 \section{Electro optic effects}%
   \label{sec:Electro optic effects}
   An electro–optic effect is a change in the optical properties of a material in response to an electric field that varies slowly compared with the frequency of light. 
   \section{Flexible energy storage devices}%
     \label{sec:Flexible energy storage devices}
Flexible energy storage devices have numerous applications, such as active radiofrequency identification tags, integrated circuit smart cards and portable electronics . The flexibility gives an added advantage as they can be embedded in tiny and flexible electronic devices.
     \section{Flexible chemical sensors}%
       \label{sec:Flexible chemical sensors}
       Traditional chemical sensors based on conductive polymers over polymer substrates are generally flexible. However, with the advent of high-quality nanowires and nanotubes, and the rapid development of flexible electronics, which involves many breakthroughs in the synthesis, and manipulation of these nanoscale materials, significant progress has been made in modern flexible chemical sensors.
       \section{Flexible Solar cells}%
         \label{sec:Flexible Solar cells}
        Flexible solar panels belong to a family of solar products called “thin film panels.” Flexible panels are constructed with silicon layers over 300 times smaller than those of standard solar panels, allowing them to be flexed and still retain their functionality. 

\end{document}
