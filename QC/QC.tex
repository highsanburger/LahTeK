\documentclass{article}
\usepackage[utf8]{inputenc}
\usepackage[margin=0.5in]{geometry}
\usepackage{amsmath}
\usepackage{amssymb}
\usepackage{physics}
\usepackage{ulem}
\usepackage{enumitem}
\usepackage{graphicx}
\usepackage{blindtext}

\usepackage{draftwatermark}
\SetWatermarkText{21BEE1288}
\SetWatermarkScale{4}
\SetWatermarkLightness{0.96}

\usepackage{pgf}
\usepackage{pgfpages}
\pgfpagesdeclarelayout{boxed}
{
  \edef\pgfpageoptionborder{0pt}
}
{
  \pgfpagesphysicalpageoptions
  {%
    logical pages=1,%
  }
  \pgfpageslogicalpageoptions{1}
  {
    border code=\pgfsetlinewidth{1pt}\pgfstroke,%
    border shrink=\pgfpageoptionborder,%
    resized width=.95\pgfphysicalwidth,%
    resized height=.95\pgfphysicalheight,%
    center=\pgfpoint{.5\pgfphysicalwidth}{.5\pgfphysicalheight}%
  }%
}
\pgfpagesuselayout{boxed}

\title{\textbf{Quantum Information and Computing Notes}}
\author{Syed Khalid}

\begin{document}

\maketitle
\section{Lecture 1}
\begin{itemize}
    \item Serial - Parallel in Classical Computing using many processors
    \item History of QC \\
    Feynman : capturing probabilistic nature of QM\\
    Shor : factorising big composite number  using QC - no classical algorithm to solven polynomial time\\
    
    \item Inherent parallelism in QC as all bits are linear combinations
    \item \textit{Entanglement} 
    \item Moore's Law - Number density of transistors doubles every two months\\
    inter transistor distance decreases - \textit{Miniaturization}\\
    as you get smaller quantum effects get more significant 
    \\ Heat produced by one component affects other \\
    Heat produced $\propto$ volume but Removing Heat $\propto$ surface
    \item  Landaur's Principle - Every physically irreversible process (most classical ) 
    $n$  bits of information increases entropy by $nk_{b} \mathrm{log} 2$
    \item Reversible classical gates - \textit{collecting garbage?}
    \item Complexity of problem - dependence of time and memory on length of input\\
    polynomial,log,constant time - easy problems\\
    $2^{n}$ - hard problem\\ 

\end{itemize}

\section{\uline{Postulates of Quantum Mechanics :-}} 
\
\begin{flushright}
\today
\end{flushright}

\begin{enumerate}
  \item 
\end{enumerate}

\section{\underline{Qubits}}
\begin{flushright}
  16/4  
\end{flushright}

\begin{itemize}
  \item cbit - 0/1 
  \item qbit - 2D Hilbert space, $\ket{0} , \ket{1} $ \\
  choose orthonormal states $  \bra{0}\ket{1} = \bra{1}\ket{0} =  0   $ and $ \bra{0}\ket{0} = \bra{1}\ket{1} = 1 $\\
  Any linear combination is valid - $\ket{\psi} = c_0 \ket{0} + c_1 \ket{1} | c_1 ,c_2 \in \mathbb{C}$ \\
  Matrix representation - 
$ ([2 3]) $
  

\end{itemize}

$\frac{\gamma}{\pi}$ 
$ k^{$1}







\end{document}