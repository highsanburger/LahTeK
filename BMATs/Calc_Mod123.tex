\documentclass{article}
\usepackage[utf8]{inputenc}
\usepackage[margin=0.5in]{geometry}
\usepackage{amsmath}
\usepackage{amssymb}
\usepackage{physics}
\usepackage{ulem}
\usepackage{enumitem}
\usepackage{graphicx}


\title{\textbf{Calculus Module 1-3 Formulae Sheet}}
\author{Syed Khalid}
\date{27/10/21}

\begin{document}

\maketitle

\uwave{\textbf{Multi Variable Calculus}}
\begin{enumerate}
    \item If $f(x)$ is continous when $x\in[a,b]$ and is differentiable when $x\in(a,b)$
        \\\textbf{Rolle's Theorem:} and if $f(a)=f(b)$ then
         $\exists c \in (a,b)$ such that $f'(c)=0$
        \\\textbf{Langrange's Mean Value Theorem:} then $\exists
         c \in (a,b)$ such that $f'(c)=\frac{f(b)-f(a)}{b-a}$
    \item \uline{First Derivative Test :} Find critical points $c$ from 
    $f'(x)=0$. $\forall \epsilon>0$ we have,\\
    If $f(x-\epsilon)<0$ and $f(x+\epsilon)>0 $ then $c$ is a local minimum.\\
    If $f(x-\epsilon)>0$ and $f(x+\epsilon)<0$ then $c$ is a local maximum.\\

    \item \uline{Second Derivative Test:} If $f'(c)=0$ \\
    and if $f''(c)>0$ then $c$ is a local minimum.\\
    and if $f''(c)<0$ then $c$ is a local maximum.\\
    and if $f''(c)=0$ and if  $\forall \epsilon>0,\,\text{sign}(f''(c-\epsilon))=-\text{sign}(f+\epsilon)$ then $c$ is an inflection point.
    
    \item Average Value of Function $f_{avg}=\frac{1}{b-a}\int_a^bf(x)dx$\\
    MVT for Integrals: if $f(x)$ is continous on $[a,b]$ then $\exists c\in[a,b]$ such that $f(c)=f_{avg}$
    
    \item Area Between Curves $A=\int^b_a\abs{f(x)-g(x)}dx$ where $\abs*{f(x)-g(x)}=\   \begin{cases}
        f(x)-g(x),& \text{when} f(x)\geq g(x)\\
        g(x)-f(x) & \text{when} g(x)\geq f(x)
    \end{cases}$ (split the integrals)
    
    \item \uline{Volumes of Solids of Revolutions:} If $f(x)\geq g(x),\, x\in [a,b]$ and $g^{-1}(y)\geq f^{-1}(y),\,y\in [c,d]$:
    \begin{itemize}
        \item About $y=q$ using disks- $I=\int_a^b\pi[(f(x)-q)^2-(g(x)-q)^2]dx$
        \item About $x=p$ using disks - $I=\int_c^d\pi[(g^{-1}(y)-p)^2-(f^{-1}(y)-p)^2]dy$
        \item About $x=p$ using shells - $I=\int_a^b2\pi|x-p|[f(x)-g(x)]dx$
        \item About $y=q$ using shells- $I=\int_c^d2\pi |y-q|[g^{-1}(y)-f^{-1}(y)]dy$

    \end{itemize}
    \item Taylor Expansion for $f(x)$ around a point $a$:
    $$f(x)=f(a)+f'(a)(x-a)+\frac{f''(a)}{2}(x-a^2)\dots\implies\frac{f(a)}{0!}(x-a)^0+\frac{f'(a)}{1!}(x-a)^1+
    \frac{f''(a)}{2!}(x-a)^2\dots=\sum_{n=0}^\infty \frac{f^{(n)}(a)}{n!}(x-a)^n$$
    For Maclaurin's series put $a=0$.
\end{enumerate}
\uwave{\textbf{Multivariable Calculus}}
\begin{enumerate}
    \item Limits of $F(x,y)$ \\ To show that limit \textit{doesnt exist} we find the limit along two different paths and find differing limit values. \\
    To \textit{prove the existence} of a limit  we use Epsilon-Delta Method: $\lim_{(x,y)\to(a,b)}f(x,y)=L$ if $\forall \epsilon>0,\,\exists \delta>0$ such that if assumed $0<\sqrt{(x-a)^2+(y-b)^2}<\delta$ then $|f(x,y)-L|<\epsilon$ (manipulate $|f(x,y)-L|$ to get inequality with $\delta$ then express $\delta$ as a function of $\epsilon$)
    \item Continuity of $F(x,y)$: $f$ is continous at point $(a,b)$ if $\lim_{(x,y)\to(a,b)}f(x,y)=f(a,b)$
    \item Partial Derivatives for a function $f(x,y)$
    \begin{enumerate}
        \item $\pdv[2]{f}{x}=\pdv{}{x}\pdv{f}{x}=f_{xx}$
        \item $\pdv{f}{x}{y}=\pdv{}{x}\pdv{f}{y}=\pdv{}{y}\pdv{f}{x}=\pdv{f}{y}{x}=f_{xy}$
        \item for $f(x(t),y(t))$: $\pdv{f}{t}=\pdv{f}{x}\dv{x}{t}+\pdv{f}{y}\dv{y}{t}$
    \end{enumerate}
    \item Total Differential Value for $z=f(x,y)$  - $dz=\pdv{f}{x}dx+\pdv{f}{y}dy$
    \item Jacobian $J=\pdv{(x,y)}{(u,v)}\implies\begin{vmatrix} \pdv{x}{u} & \pdv{x}{v} \\ \pdv{y}{u} & \pdv{y}{v} \end{vmatrix}=\pdv{x}{u}\pdv{y}{v}-\pdv{x}{v}\pdv{y}{u}$

    
     Also if $J^{-1}=\pdv{(u,v)}{(x,y)}$ we have, $J^{-1}J=1$
    \\ Two functions $u(x,y),v(x,y)$ are functionally dependent iff $\pdv{(u,v)}{(x,y)}=0$

\end{enumerate} 
\uwave{\textbf{Application of Multivariable Calculus  }}
\begin{enumerate}
    \item Taylor Expansion for two variables for approximating $f$ around the point $(a,b)$:
    $$f(x,y)=\sum_{i=0}^n\sum_{j=0}^{n-i}\frac{1}{i!j!}\frac{\partial^{(i+j)}f(a,b)}{\partial x^i \partial y^j}(x-a)^i(y-b)^j$$
    First Order Approximation ($n=1$):$f(x,y)\approx L(x,y)= f(a,b)+f_x(a,b)(x-a)+f_y(a,b)(y-b)$
    \\Second Order Approximation ($n=2$):$f \approx L(x,y)+\frac{1}{2}f_{xx}(a,b)(x-a)^2+f_{xy}(a,b)(x-a)(y-b)+\frac{1}{2}f_{yy}(a,b)(y-b)^2$
    \item For Maxima and Minima of $f(x,y)$, find stationary points from the equations $f_x=0$ and $f_y=0$. 
    Then evaluate $f_{xx},f_{yy},f_{xy}$ at stationary point $P(a,b)$
    \begin{itemize}
        \item If $f_{xx}f_{yy}>(f_{xy})^2$ and $f_{xx}<0$ or $f_{yy}<0$ then $P$ is a local Maxima
        \item If $f_{xx}f_{yy}>(f_{xy})^2$ and $f_{xx}>0$ or $f_{yy}>0$ then $P$ is a local Minima
        \item If $f_{xx}f_{yy}<(f_{xy})^2$ then $P$ is a saddle point(neither a minima nor a maxima). 
        \item If $f_{xx}f_{yy}=(f_{xy})^2$ then anything is possible (\textit{RIP!}).
    \end{itemize}
    \item Constrained Maxima and Minima of $F(x,y)$
    \item Langrange's Multiplier Method
    
\end{enumerate}

    
\end{document}
